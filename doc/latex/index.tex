\hyperlink{classPartsBasedDetector}{\-Parts\-Based\-Detector} is a visual object recognition technique described in the following series of papers by \-Deva \-Ramanan\-:


\begin{DoxyItemize}
\item \-X. \-Zhu and \-D. \-Ramanan. \char`\"{}\-Face detection, pose estimation and landmark localization
 in the wild\char`\"{}, \-Internation \-Conference on \-Computer \-Vision and \-Pattern \-Recognition (\-C\-V\-P\-R), 2012
\end{DoxyItemize}


\begin{DoxyItemize}
\item \-Y. \-Yang and \-D. \-Ramanan. \char`\"{}\-Articulated pose estimation using flexible mixtures of parts\char`\"{}, \-International \-Conference on \-Computer \-Vision and \-Pattern \-Recognition (\-C\-V\-P\-R), 2011
\end{DoxyItemize}


\begin{DoxyItemize}
\item \-P. \-Felzenszwalb, \-R. \-Girshick, \-D. \-Mc\-Allester and \-D. \-Ramanan. \char`\"{}\-Object detection with
 disciminatively trained part based models\char`\"{}, \-Journal on \-Pattern \-Analysis and \-Machine \-Intelligence (\-P\-A\-M\-I), 2010
\end{DoxyItemize}

\-Holistic appearance of complex deformations are difficult to model directly, so parts based models break a holistic detector into a series of smaller \char`\"{}part\char`\"{} detectors with geometric constraints enforcing particular spatial relationships between the parts. \-In this manner, deformation of individual parts can be assumed to be linear, and the geometric constraints can be approximated by a linear subspace, spring-\/mass damper system, etc.

\hyperlink{classPartsBasedDetector}{\-Parts\-Based\-Detector} can detect many types of objects including\-:
\begin{DoxyItemize}
\item \-Rigid objects (coffee mugs)
\item \-Objects of common semantic class but different appearance (chairs)
\item \-Deformable objects (faces)
\item \-Articulated objects (human bodies)
\end{DoxyItemize}

\-The model supports three main features\-:
\begin{DoxyItemize}
\item search across scale
\item multiple components (different views of an object where a different subset of parts might be visible in each view, frontal and profile faces for example)
\item multiple mixtures (each part may have multiple components or \char`\"{}views\char`\"{} such as a closed vs open hand, or a vertically oriented vs a horizontally oriented hand
\end{DoxyItemize}

\-The capacity of the classifier is largely a product of the training data supplied

\-Detecting objects using the \hyperlink{classPartsBasedDetector}{\-Parts\-Based\-Detector} requires three main steps\-:
\begin{DoxyItemize}
\item \-Instantiation of model classes
\item \-Detection
\item \-Visualization
\end{DoxyItemize}

\-The following example code shows a common way to perform this pipeline\-: 
\begin{DoxyCode}
 // create the model object and deserialize it
 MatlabIOModel model;
 model.deserialize(argv[1]);

 // create the PartsBasedDetector and distribute the model parameters
 PartsBasedDetector<double> pbd;
 pbd.distributeModel(model);

 // load the image from file
 Mat im = imread(argv[2]);

 // detect potential candidates in the image
 vector<Candidate> candidates;
 pbd.detect(im, candidates);

 // visualize the best 5 detection candidates
 Visualize visualize(model.name());
 if (candidates.size() > 0) {
  Mat canvas;
        Candidate::sort(candidates);
        visualize.candidates(im, candidates, 5, canvas, true);
        visualize.image(canvas);
        waitKey();
 }
\end{DoxyCode}


\-And that's all there is to it!

\-Multiple pre-\/trained models are available. \-Currently included are\-:
\begin{DoxyItemize}
\item \-Human body detector
\item \-Face detector
\end{DoxyItemize}

\hyperlink{classModel}{\-Model} training is currently only supported via \-Matlab code provided by \-Deva \-Ramanan. \hyperlink{classMatlabIOModel}{\-Matlab\-I\-O\-Model} provides a method for deserializing models generated by \-Matlab and saved in the .\-Mat format.

-\/-\/-\/-\/-\/-\/-\/-\/-\/-\/

\-This packaged is written and maintained by \-Hilton \-Bristow, \-Willow \-Garage with the consent of \-Deva \-Ramanan. \-The package is released under a \-B\-S\-D license. \-Please see the included license file for details and acknowledgement of contributions. 